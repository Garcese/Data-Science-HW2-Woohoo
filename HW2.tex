\documentclass{article}\usepackage[]{graphicx}\usepackage[]{color}
% maxwidth is the original width if it is less than linewidth
% otherwise use linewidth (to make sure the graphics do not exceed the margin)
\makeatletter
\def\maxwidth{ %
  \ifdim\Gin@nat@width>\linewidth
    \linewidth
  \else
    \Gin@nat@width
  \fi
}
\makeatother

\definecolor{fgcolor}{rgb}{0.345, 0.345, 0.345}
\newcommand{\hlnum}[1]{\textcolor[rgb]{0.686,0.059,0.569}{#1}}%
\newcommand{\hlstr}[1]{\textcolor[rgb]{0.192,0.494,0.8}{#1}}%
\newcommand{\hlcom}[1]{\textcolor[rgb]{0.678,0.584,0.686}{\textit{#1}}}%
\newcommand{\hlopt}[1]{\textcolor[rgb]{0,0,0}{#1}}%
\newcommand{\hlstd}[1]{\textcolor[rgb]{0.345,0.345,0.345}{#1}}%
\newcommand{\hlkwa}[1]{\textcolor[rgb]{0.161,0.373,0.58}{\textbf{#1}}}%
\newcommand{\hlkwb}[1]{\textcolor[rgb]{0.69,0.353,0.396}{#1}}%
\newcommand{\hlkwc}[1]{\textcolor[rgb]{0.333,0.667,0.333}{#1}}%
\newcommand{\hlkwd}[1]{\textcolor[rgb]{0.737,0.353,0.396}{\textbf{#1}}}%
\let\hlipl\hlkwb

\usepackage{framed}
\makeatletter
\newenvironment{kframe}{%
 \def\at@end@of@kframe{}%
 \ifinner\ifhmode%
  \def\at@end@of@kframe{\end{minipage}}%
  \begin{minipage}{\columnwidth}%
 \fi\fi%
 \def\FrameCommand##1{\hskip\@totalleftmargin \hskip-\fboxsep
 \colorbox{shadecolor}{##1}\hskip-\fboxsep
     % There is no \\@totalrightmargin, so:
     \hskip-\linewidth \hskip-\@totalleftmargin \hskip\columnwidth}%
 \MakeFramed {\advance\hsize-\width
   \@totalleftmargin\z@ \linewidth\hsize
   \@setminipage}}%
 {\par\unskip\endMakeFramed%
 \at@end@of@kframe}
\makeatother

\definecolor{shadecolor}{rgb}{.97, .97, .97}
\definecolor{messagecolor}{rgb}{0, 0, 0}
\definecolor{warningcolor}{rgb}{1, 0, 1}
\definecolor{errorcolor}{rgb}{1, 0, 0}
\newenvironment{knitrout}{}{} % an empty environment to be redefined in TeX

\usepackage{alltt}
\usepackage{amsmath} %This allows me to use the align functionality.
                     %If you find yourself trying to replicate
                     %something you found online, ensure you're
                     %loading the necessary packages!
\usepackage{amsfonts}%Math font
\usepackage{graphicx}%For including graphics
\usepackage{hyperref}%For Hyperlinks
\hypersetup{colorlinks = true,citecolor=black}
\usepackage{natbib}        %For the bibliography
\bibliographystyle{apalike}%For the bibliography
\usepackage[margin=1.0in]{geometry}
\usepackage{float}
\IfFileExists{upquote.sty}{\usepackage{upquote}}{}
\begin{document}
\noindent \textbf{MA 354: Data Analysis -- Fall 2021 -- Due 10/8 at 5p}\\%\\ gives you a new line
\noindent \textbf{Homework 2:}\vspace{1em}\\
\emph{Complete the following opportunities to use what we've talked about in class. These questions will be graded for correctness, communication and succinctness. Ensure you show your work and explain your logic in a legible and refined submission.}\\\vspace{1em}
%Comments -- anything after % is not put into the PDF

The starting jobs will be applied in alphabetical order (last name) for question two.
\begin{enumerate}
  \item \textbf{Solver:} provide a solution, if possible, and reasoning for the solution. \textbf{Due to group 10/5 or earlier.}
  \item \textbf{Code Checker:} provides a first check of the solver's worked solutions and ensures they are correct with a solid interpretation. 
  \item \textbf{Checker} checks the solution for completeness, proposes and implements changes if agreed upon by the group. Provides a short paragraph summarizing the discussion of proposals and their reason for acceptance or non-acceptance.
  \item \textbf{Double Checker} checks the solution for completeness, communication and polish. The Double Checker ensures that the solution is correct and highly polished for submission.
\end{enumerate}

\noindent For subsequent questions student roles will move down one position. The rolls change as follows.
\begin{enumerate}
  \item \textbf{Solver} $\Longrightarrow$ \textbf{Code Checker}
  \item \textbf{Code Checker} $\Longrightarrow$ \textbf{Checker}
  \item \textbf{Checker} $\Longrightarrow$ \textbf{Double Checker}
  \item \textbf{Double Checker} $\Longrightarrow$ \textbf{Solver}
\end{enumerate}
While students have assigned jobs for each question I encourage students to help 
each other complete the homework in collaboration.
\newpage
\begin{enumerate}
%%%%%%%%%%%%%%%%%%%%%%%%%%%%%%%%%%%%%%%%%%%%%%%%%%%%%%%%%%%%%%%%%%%%%%%%%%%%%%%%%%%%%%%%%%%
%%%%%%%%%%%%%%%%%%%%%%%%%%%%%%%%%%%%%%%%%%%%%%%%%%%%%%%%%%%%%%%%%%%%%%%%%%%%%%%%%%%%%%%%%%%
%%%%%%%%%  Question 1
%%%%%%%%%%%%%%%%%%%%%%%%%%%%%%%%%%%%%%%%%%%%%%%%%%%%%%%%%%%%%%%%%%%%%%%%%%%%%%%%%%%%%%%%%%%
%%%%%%%%%%%%%%%%%%%%%%%%%%%%%%%%%%%%%%%%%%%%%%%%%%%%%%%%%%%%%%%%%%%%%%%%%%%%%%%%%%%%%%%%%%%
  \item\label{Q1} Select a continuous distribution (Not the uniform or exponential). 
  It does not have to be one that we cover in the notes! To explore the PDF of your 
  distribution, specify two sets of parameter(s) for your distribution.
  \begin{enumerate}
  %%%%%%%%%%%%%%%%%%%%%%%%%%%%%%%%%%%%%%%%%%%%%%%%%%%%%%%%%%%%%%%%%%%%%%%%%%%%%%%%%%%%%%%%%%%
  %%%%%%%%%  Part (a)
  %%%%%%%%%%%%%%%%%%%%%%%%%%%%%%%%%%%%%%%%%%%%%%%%%%%%%%%%%%%%%%%%%%%%%%%%%%%%%%%%%%%%%%%%%%%
  \item \textbf{History} Discuss what types of random variables are modeled with 
  your distribution. Be sure to include a discussion about the support and ensure 
  to provide the density function, and CDF. This requires some internet research 
  -- what's the history of the distribution, why was it created and named? What 
  are some exciting applications of this distribution?
  
  Cite all of your sources in LaTeX by adding a BibTeX citation to the .bib file. 
  To help, I've cited R \citep{R21} in parentheses here. \cite{R21} provides helpful 
  tools for the rest of the questions below. BibTeX citations are available through 
  Google Scholar by clicking the cite button below the article of  interest and 
  selecting the BibTeX option.
  %%%%%%%%%%%%%%%%%%%%%%%%%%%%%%%%%%%%%%%%%%%%%%%%%%%%%%%%%%%%%%%%%%%%%%%%%%%%%%%%%%%%%%%%%%%
  %%%%%%%%%  Part (b)
  %%%%%%%%%%%%%%%%%%%%%%%%%%%%%%%%%%%%%%%%%%%%%%%%%%%%%%%%%%%%%%%%%%%%%%%%%%%%%%%%%%%%%%%%%%%
	\item Show that you have a valid PDF. You will find the \texttt{integrate()} 
	function in \texttt{R} helpful.
	%%%%%%%%%%%%%%%%%%%%%%%%%%%%%%%%%%%%%%%%%%%%%%%%%%%%%%%%%%%%%%%%%%%%%%%%%%%%%%%%%%%%%%%%%%%
  %%%%%%%%%  Part (c)
  %%%%%%%%%%%%%%%%%%%%%%%%%%%%%%%%%%%%%%%%%%%%%%%%%%%%%%%%%%%%%%%%%%%%%%%%%%%%%%%%%%%%%%%%%%%
	\item Find the median for your two sets of parameter(s). Conduct some research 
	to find the median based on our PDF to confirm that your numerical approach is 
	correct. 
	%%%%%%%%%%%%%%%%%%%%%%%%%%%%%%%%%%%%%%%%%%%%%%%%%%%%%%%%%%%%%%%%%%%%%%%%%%%%%%%%%%%%%%%%%%%
  %%%%%%%%%  Part (d)
  %%%%%%%%%%%%%%%%%%%%%%%%%%%%%%%%%%%%%%%%%%%%%%%%%%%%%%%%%%%%%%%%%%%%%%%%%%%%%%%%%%%%%%%%%%%
	\item \label{q1PDF} Graph the PDF for several values of the parameter(s) 
	including the two sets you specified. What does changing the parameter(s) do 
	to the shape of the PDF?
	%%%%%%%%%%%%%%%%%%%%%%%%%%%%%%%%%%%%%%%%%%%%%%%%%%%%%%%%%%%%%%%%%%%%%%%%%%%%%%%%%%%%%%%%%%%
  %%%%%%%%%  Part (e)
  %%%%%%%%%%%%%%%%%%%%%%%%%%%%%%%%%%%%%%%%%%%%%%%%%%%%%%%%%%%%%%%%%%%%%%%%%%%%%%%%%%%%%%%%%%%
	 \item Graph the CDF for the same values of the parameter(s) as you did in 
	 Question \ref{q1PDF}. What does changing the parameter(s) do to the shape of 
	 the CDF? Comment on the aspects of the CDFs that show that the CDF is valid.
	%%%%%%%%%%%%%%%%%%%%%%%%%%%%%%%%%%%%%%%%%%%%%%%%%%%%%%%%%%%%%%%%%%%%%%%%%%%%%%%%%%%%%%%%%%%
  %%%%%%%%%  Part (f)
  %%%%%%%%%%%%%%%%%%%%%%%%%%%%%%%%%%%%%%%%%%%%%%%%%%%%%%%%%%%%%%%%%%%%%%%%%%%%%%%%%%%%%%%%%%%
  \item Generate a random sample of size $n=10, 25, 100$, and $1000$ for your 
  two sets of parameter(s). In a $4 \times 2$ grid, plot a histogram of each set
  of data and superimpose the true density function at the specified parameter 
  values. Interpret the results.
	\end{enumerate}
%%%%%%%%%%%%%%%%%%%%%%%%%%%%%%%%%%%%%%%%%%%%%%%%%%%%%%%%%%%%%%%%%%%%%%%%%%%%%%%%%%%%%%%%%%%
%%%%%%%%%%%%%%%%%%%%%%%%%%%%%%%%%%%%%%%%%%%%%%%%%%%%%%%%%%%%%%%%%%%%%%%%%%%%%%%%%%%%%%%%%%%
%%%%%%%%%  Question 2
%%%%%%%%%%%%%%%%%%%%%%%%%%%%%%%%%%%%%%%%%%%%%%%%%%%%%%%%%%%%%%%%%%%%%%%%%%%%%%%%%%%%%%%%%%%
%%%%%%%%%%%%%%%%%%%%%%%%%%%%%%%%%%%%%%%%%%%%%%%%%%%%%%%%%%%%%%%%%%%%%%%%%%%%%%%%%%%%%%%%%%%
\item Continue with the continuous distribution you selected for Question \ref{Q1}.
\begin{enumerate}
  %%%%%%%%%%%%%%%%%%%%%%%%%%%%%%%%%%%%%%%%%%%%%%%%%%%%%%%%%%%%%%%%%%%%%%%%%%%%%%%%%%%%%%%%%%%
  %%%%%%%%%  Part (a)
  %%%%%%%%%%%%%%%%%%%%%%%%%%%%%%%%%%%%%%%%%%%%%%%%%%%%%%%%%%%%%%%%%%%%%%%%%%%%%%%%%%%%%%%%%%%
  \item Provide the mean, standard deviation, skewness, and kurtosis of the PDF.
  Ensure to interpret each.
  %%%%%%%%%%%%%%%%%%%%%%%%%%%%%%%%%%%%%%%%%%%%%%%%%%%%%%%%%%%%%%%%%%%%%%%%%%%%%%%%%%%%%%%%%%%
  %%%%%%%%%  Part (b)
  %%%%%%%%%%%%%%%%%%%%%%%%%%%%%%%%%%%%%%%%%%%%%%%%%%%%%%%%%%%%%%%%%%%%%%%%%%%%%%%%%%%%%%%%%%%
  \item Generate a random sample of size $n=10, 25, 100$, and $1000$ for your 
  two sets of parameter(s). Calculate the sample mean, standard deviation, 
  skewness, and kurtosis. Interpret the results.
  %%%%%%%%%%%%%%%%%%%%%%%%%%%%%%%%%%%%%%%%%%%%%%%%%%%%%%%%%%%%%%%%%%%%%%%%%%%%%%%%%%%%%%%%%%%
  %%%%%%%%%  Part (c)
  %%%%%%%%%%%%%%%%%%%%%%%%%%%%%%%%%%%%%%%%%%%%%%%%%%%%%%%%%%%%%%%%%%%%%%%%%%%%%%%%%%%%%%%%%%%
  \item Generate a random sample of size $n=10$ for your two sets of parameter(s).
  Calculate the method of moments estimator(s) and maximum likelihood estimator(s).
  In a $1 \times 2$ grid, plot a histogram of each set of data with (1) the method 
  of moments estimated distribution, (2) the maximum likelihood estimated 
  distribution, and superimpose the true distribution in both.
  %%%%%%%%%%%%%%%%%%%%%%%%%%%%%%%%%%%%%%%%%%%%%%%%%%%%%%%%%%%%%%%%%%%%%%%%%%%%%%%%%%%%%%%%%%%
  %%%%%%%%%  Part (d)
  %%%%%%%%%%%%%%%%%%%%%%%%%%%%%%%%%%%%%%%%%%%%%%%%%%%%%%%%%%%%%%%%%%%%%%%%%%%%%%%%%%%%%%%%%%%
  \item Generate a random sample of size $n=25$ for your two sets of parameter(s).
  Calculate the method of moments estimator(s) and maximum likelihood estimator(s). 
  In a $1 \times 2$ grid, plot a histogram of each set of data with (1) the method 
  of moments estimated distribution, (2) the maximum likelihood estimated distribution, 
  and superimpose the true distribution in both.
  %%%%%%%%%%%%%%%%%%%%%%%%%%%%%%%%%%%%%%%%%%%%%%%%%%%%%%%%%%%%%%%%%%%%%%%%%%%%%%%%%%%%%%%%%%%
  %%%%%%%%%  Part (e)
  %%%%%%%%%%%%%%%%%%%%%%%%%%%%%%%%%%%%%%%%%%%%%%%%%%%%%%%%%%%%%%%%%%%%%%%%%%%%%%%%%%%%%%%%%%%
  \item Generate a random sample of size $n=100$ for your two sets of parameter(s). 
  Calculate the method of moments estimator(s) and maximum likelihood estimator(s).
  In a $1 \times 2$ grid, plot a histogram of each set of data with (1) the method 
  of moments estimated distribution, (2) the maximum likelihood estimated distribution,
  and superimpose the true distribution in both.
  %%%%%%%%%%%%%%%%%%%%%%%%%%%%%%%%%%%%%%%%%%%%%%%%%%%%%%%%%%%%%%%%%%%%%%%%%%%%%%%%%%%%%%%%%%%
  %%%%%%%%%  Part (f)
  %%%%%%%%%%%%%%%%%%%%%%%%%%%%%%%%%%%%%%%%%%%%%%%%%%%%%%%%%%%%%%%%%%%%%%%%%%%%%%%%%%%%%%%%%%%
  \item Generate a random sample of size $n=100$ for your two sets of parameter(s). 
  Calculate the method of moments estimator(s) and maximum likelihood estimator(s). 
  In a $1 \times 2$ grid, plot a histogram of each set of data with (1) the method 
  of moments estimated distribution, (2) the maximum likelihood estimated distribution, 
  and superimpose the true distribution in both.
  %%%%%%%%%%%%%%%%%%%%%%%%%%%%%%%%%%%%%%%%%%%%%%%%%%%%%%%%%%%%%%%%%%%%%%%%%%%%%%%%%%%%%%%%%%%
  %%%%%%%%%  Part (g)
  %%%%%%%%%%%%%%%%%%%%%%%%%%%%%%%%%%%%%%%%%%%%%%%%%%%%%%%%%%%%%%%%%%%%%%%%%%%%%%%%%%%%%%%%%%%
  \item Comment on the results of parts (c)-(f). 
\end{enumerate}
\newpage
%%%%%%%%%%%%%%%%%%%%%%%%%%%%%%%%%%%%%%%%%%%%%%%%%%%%%%%%%%%%%%%%%%%%%%%%%%%%%%%%%%%%%%%%%%%
%%%%%%%%%%%%%%%%%%%%%%%%%%%%%%%%%%%%%%%%%%%%%%%%%%%%%%%%%%%%%%%%%%%%%%%%%%%%%%%%%%%%%%%%%%%
%%%%%%%%%  Question 3
%%%%%%%%%%%%%%%%%%%%%%%%%%%%%%%%%%%%%%%%%%%%%%%%%%%%%%%%%%%%%%%%%%%%%%%%%%%%%%%%%%%%%%%%%%%
%%%%%%%%%%%%%%%%%%%%%%%%%%%%%%%%%%%%%%%%%%%%%%%%%%%%%%%%%%%%%%%%%%%%%%%%%%%%%%%%%%%%%%%%%%%
  \item\label{Q3} Select a discrete distribution (not the Poisson). It does not 
  have to be one that we cover in the notes! To explore the PMF of your distribution, 
  specify two sets of parameter(s) for your distribution.
  \begin{enumerate}
  %%%%%%%%%%%%%%%%%%%%%%%%%%%%%%%%%%%%%%%%%%%%%%%%%%%%%%%%%%%%%%%%%%%%%%%%%%%%%%%%%%%%%%%%%%%
  %%%%%%%%%  Part (a)
  %%%%%%%%%%%%%%%%%%%%%%%%%%%%%%%%%%%%%%%%%%%%%%%%%%%%%%%%%%%%%%%%%%%%%%%%%%%%%%%%%%%%%%%%%%%
  \item \textbf{History} Discuss what types of random variables are modeled with 
  your distribution. Be sure to include a discussion about the support and ensure
  to provide the mass function, and CDF. This requires some internet research -- 
  what's the history of the distribution, why was it created and named? What are
  some exciting applications of this distribution? Cite all of your sources.
  %%%%%%%%%%%%%%%%%%%%%%%%%%%%%%%%%%%%%%%%%%%%%%%%%%%%%%%%%%%%%%%%%%%%%%%%%%%%%%%%%%%%%%%%%%%
  %%%%%%%%%  Part (b)
  %%%%%%%%%%%%%%%%%%%%%%%%%%%%%%%%%%%%%%%%%%%%%%%%%%%%%%%%%%%%%%%%%%%%%%%%%%%%%%%%%%%%%%%%%%%
	\item Show that you have a valid PMF. You can show this approximately by 
	calculating the series in a repeat loop until probability mass evaluations are 
	infinitesimally small.
	%%%%%%%%%%%%%%%%%%%%%%%%%%%%%%%%%%%%%%%%%%%%%%%%%%%%%%%%%%%%%%%%%%%%%%%%%%%%%%%%%%%%%%%%%%%
  %%%%%%%%%  Part (c)
  %%%%%%%%%%%%%%%%%%%%%%%%%%%%%%%%%%%%%%%%%%%%%%%%%%%%%%%%%%%%%%%%%%%%%%%%%%%%%%%%%%%%%%%%%%%
	\item Find the median for your two sets of parameter(s). Conduct some research 
	to find the median based on our PMF to confirm that your numerical approach is
	correct. 
	%%%%%%%%%%%%%%%%%%%%%%%%%%%%%%%%%%%%%%%%%%%%%%%%%%%%%%%%%%%%%%%%%%%%%%%%%%%%%%%%%%%%%%%%%%%
  %%%%%%%%%  Part (d)
  %%%%%%%%%%%%%%%%%%%%%%%%%%%%%%%%%%%%%%%%%%%%%%%%%%%%%%%%%%%%%%%%%%%%%%%%%%%%%%%%%%%%%%%%%%%
	\item \label{q3PMF} Graph the PMF for several values of the parameter(s) 
	including the two sets you specified. What does changing the parameter(s) do 
	to the shape of the PMF?
	%%%%%%%%%%%%%%%%%%%%%%%%%%%%%%%%%%%%%%%%%%%%%%%%%%%%%%%%%%%%%%%%%%%%%%%%%%%%%%%%%%%%%%%%%%%
  %%%%%%%%%  Part (e)
  %%%%%%%%%%%%%%%%%%%%%%%%%%%%%%%%%%%%%%%%%%%%%%%%%%%%%%%%%%%%%%%%%%%%%%%%%%%%%%%%%%%%%%%%%%%
	 \item Graph the CDF for the same values of the parameter(s) as you did in 
	 Question \ref{q3PMF}. What does changing the parameter(s) do to the shape of 
	 the CDF? Comment on the aspects of the CDFs that show that the CDF is valid.
	%%%%%%%%%%%%%%%%%%%%%%%%%%%%%%%%%%%%%%%%%%%%%%%%%%%%%%%%%%%%%%%%%%%%%%%%%%%%%%%%%%%%%%%%%%%
  %%%%%%%%%  Part (f)
  %%%%%%%%%%%%%%%%%%%%%%%%%%%%%%%%%%%%%%%%%%%%%%%%%%%%%%%%%%%%%%%%%%%%%%%%%%%%%%%%%%%%%%%%%%%
  \item Generate a random sample of size $n=10, 25, 100$, and $1000$ for your 
  two sets of parameter(s). In a $4 \times 2$ grid, plot a histogram (with bin 
  size 1) of each set of data and superimpose the true mass function at the 
  specified parameter values. Interpret the results.
	\end{enumerate}
%%%%%%%%%%%%%%%%%%%%%%%%%%%%%%%%%%%%%%%%%%%%%%%%%%%%%%%%%%%%%%%%%%%%%%%%%%%%%%%%%%%%%%%%%%%
%%%%%%%%%%%%%%%%%%%%%%%%%%%%%%%%%%%%%%%%%%%%%%%%%%%%%%%%%%%%%%%%%%%%%%%%%%%%%%%%%%%%%%%%%%%
%%%%%%%%%  Question 2
%%%%%%%%%%%%%%%%%%%%%%%%%%%%%%%%%%%%%%%%%%%%%%%%%%%%%%%%%%%%%%%%%%%%%%%%%%%%%%%%%%%%%%%%%%%
%%%%%%%%%%%%%%%%%%%%%%%%%%%%%%%%%%%%%%%%%%%%%%%%%%%%%%%%%%%%%%%%%%%%%%%%%%%%%%%%%%%%%%%%%%%
\item Continue with the discrete distribution you selected for Question \ref{Q3}.
\begin{enumerate}
  %%%%%%%%%%%%%%%%%%%%%%%%%%%%%%%%%%%%%%%%%%%%%%%%%%%%%%%%%%%%%%%%%%%%%%%%%%%%%%%%%%%%%%%%%%%
  %%%%%%%%%  Part (a)
  %%%%%%%%%%%%%%%%%%%%%%%%%%%%%%%%%%%%%%%%%%%%%%%%%%%%%%%%%%%%%%%%%%%%%%%%%%%%%%%%%%%%%%%%%%%
  \item Provide the mean, standard deviation, skewness, and kurtosis of the PMF. 
  Ensure to interpret each.\\
  \textbf{Solution:}\\
  The binomial distribution helps model discrete sets of data with a certain 
  number of trials with a probability of success. 
  The mean of this distribution is 
  \[\textbf{Mean} = np\]
  This makes sense, because the average number of successes is given by the 
  above formula.
  The standard deviation of this distribution is 
  \[\textbf{SD} = np(1-p)\]
  This is just multiplying the probability of failiure with the mean, which then
  gives the variance. 
  The skewness is 
  \[\textbf{Skewness} = \frac{1 - 2p}{\sqrt{np(1-p)}}\]
  For a small p and small n, the distribution will be skewed right since there 
  are a low number of successes. For a large p and small n, the distribution is
  skewed left, since there are a high number of successes. For p = 0.5, the 
  skewness is 0, since the distribution becomes symmetric. 
  The kurtosis of the distribution is 
  \[\textbf{Kurtosis} = \frac{1-6p(1-p)}{np(1-p)}\]
  The kurtosis tells us the rate of outliers in a given set of data. 
  %%%%%%%%%%%%%%%%%%%%%%%%%%%%%%%%%%%%%%%%%%%%%%%%%%%%%%%%%%%%%%%%%%%%%%%%%%%%%%%%%%%%%%%%%%%
  %%%%%%%%%  Part (b)
  %%%%%%%%%%%%%%%%%%%%%%%%%%%%%%%%%%%%%%%%%%%%%%%%%%%%%%%%%%%%%%%%%%%%%%%%%%%%%%%%%%%%%%%%%%%
  \item Generate a random sample of size $n=10, 25, 100$, and $1000$ for your 
  two sets of parameter(s). Calculate the sample mean, standard deviation, 
  skewness, and kurtosis. Interpret the results.\\
  \textbf{Solution:}
\begin{knitrout}
\definecolor{shadecolor}{rgb}{0.969, 0.969, 0.969}\color{fgcolor}\begin{kframe}
\begin{alltt}
\hlkwd{library}\hlstd{(e1071)} \hlcom{#allows skewness and kurtosis}
\hlcom{#The two sets of parameters are -}
\hlcom{#40 coin tosses with the success being a heads }
\hlkwd{set.seed}\hlstd{(}\hlnum{345}\hlstd{)} \hlcom{#helps with testing again}
\hlstd{t1} \hlkwb{=} \hlnum{40}
\hlstd{prob1} \hlkwb{=} \hlnum{0.5}
\hlcom{#20 rolls of a D4 with success being a 1 rolled}
\hlstd{t2} \hlkwb{=} \hlnum{20}
\hlstd{prob2} \hlkwb{=} \hlnum{0.25}
\hlcom{#generating each set of data for each parameter}
\hlstd{n.10.1} \hlkwb{<-} \hlkwd{rbinom}\hlstd{(}\hlnum{10}\hlstd{, t1, prob1)}
\hlstd{n.10.2} \hlkwb{<-} \hlkwd{rbinom}\hlstd{(}\hlnum{10}\hlstd{, t2, prob2)}
\hlstd{n.25.1} \hlkwb{<-} \hlkwd{rbinom}\hlstd{(}\hlnum{25}\hlstd{, t1, prob1)}
\hlstd{n.25.2} \hlkwb{<-} \hlkwd{rbinom}\hlstd{(}\hlnum{25}\hlstd{, t2, prob2)}
\hlstd{n.100.1} \hlkwb{<-} \hlkwd{rbinom}\hlstd{(}\hlnum{100}\hlstd{, t1, prob1)}
\hlstd{n.100.2} \hlkwb{<-} \hlkwd{rbinom}\hlstd{(}\hlnum{100}\hlstd{, t2, prob2)}
\hlstd{n.1000.1} \hlkwb{<-} \hlkwd{rbinom}\hlstd{(}\hlnum{1000}\hlstd{, t1, prob1)}
\hlstd{n.1000.2} \hlkwb{<-} \hlkwd{rbinom}\hlstd{(}\hlnum{1000}\hlstd{, t2, prob2)}
\hlcom{#making a dataframe of the samples}
\hlstd{stat1} \hlkwb{<-} \hlkwd{data.frame}\hlstd{(n.10.1,}
                    \hlstd{n.25.1,}
                    \hlstd{n.100.1,}
                    \hlstd{n.1000.1,}
                    \hlstd{n.10.2,}
                    \hlstd{n.25.2,}
                    \hlstd{n.100.2,}
                    \hlstd{n.1000.2)}
\hlcom{#applying the R functions and listing the results}
\hlcom{#Mean}
\hlkwd{apply}\hlstd{(stat1,} \hlnum{2}\hlstd{, mean)}
\end{alltt}
\begin{verbatim}
##   n.10.1   n.25.1  n.100.1 n.1000.1   n.10.2   n.25.2  n.100.2 n.1000.2 
##    19.60    20.52    19.84    19.88     5.90     4.28     4.98     5.01
\end{verbatim}
\begin{alltt}
\hlcom{#Standard Deviation}
\hlkwd{apply}\hlstd{(stat1,} \hlnum{2}\hlstd{, sd)}
\end{alltt}
\begin{verbatim}
##   n.10.1   n.25.1  n.100.1 n.1000.1   n.10.2   n.25.2  n.100.2 n.1000.2 
## 2.108185 3.603135 3.121561 3.203102 2.344247 1.662638 1.828372 1.886653
\end{verbatim}
\begin{alltt}
\hlcom{#Skewness}
\hlkwd{apply}\hlstd{(stat1,} \hlnum{2}\hlstd{, skewness)}
\end{alltt}
\begin{verbatim}
##      n.10.1      n.25.1     n.100.1    n.1000.1      n.10.2      n.25.2 
##  0.27663605 -0.23458002  0.22147552  0.08987464 -0.29900331  0.59578386 
##     n.100.2    n.1000.2 
##  0.02951240  0.19109951
\end{verbatim}
\begin{alltt}
\hlcom{#kurtosis}
\hlkwd{apply}\hlstd{(stat1,} \hlnum{2}\hlstd{, kurtosis)}
\end{alltt}
\begin{verbatim}
##      n.10.1      n.25.1     n.100.1    n.1000.1      n.10.2      n.25.2 
## -0.70875300 -0.49916581  0.01274940 -0.05466433 -1.40963737  0.94517654 
##     n.100.2    n.1000.2 
## -0.25228810 -0.04125743
\end{verbatim}
\end{kframe}
\end{knitrout}
  The mean is usually close to the the actual mean regardless of number of 
  samples. It would make sense for the standard deviation to increase with the
  sample size, since there are more values. 
  The skewness and kurtosis both tend to decreases with a higher number of 
  samples. 
  %%%%%%%%%%%%%%%%%%%%%%%%%%%%%%%%%%%%%%%%%%%%%%%%%%%%%%%%%%%%%%%%%%%%%%%%%%%%%%%%%%%%%%%%%%%
  %%%%%%%%%  Part (c)
  %%%%%%%%%%%%%%%%%%%%%%%%%%%%%%%%%%%%%%%%%%%%%%%%%%%%%%%%%%%%%%%%%%%%%%%%%%%%%%%%%%%%%%%%%%%
  \item Generate a random sample of size $n=10$ for your two sets of parameter(s).
  Calculate the method of moments estimator(s) and maximum likelihood estimator(s).
  In a $1 \times 2$ grid, plot a histogram (with bin size 1) of each set of data 
  with (1) the method of moments estimated distribution, (2) the maximum likelihood 
  estimated distribution, and superimpose the true distribution in both.\\
   \textbf{Solution: }
   The functions that are used for parts c through f are listed below.
   The moments for the binomial distribution were obtained via \citep{binomial2002},
   and
\begin{knitrout}
\definecolor{shadecolor}{rgb}{0.969, 0.969, 0.969}\color{fgcolor}\begin{kframe}
\begin{alltt}
\hlkwd{set.seed}\hlstd{(}\hlnum{32423}\hlstd{)}
\hlkwd{library}\hlstd{(tidyverse)}
\hlkwd{library}\hlstd{(patchwork)}
\hlstd{binom.mom} \hlkwb{<-} \hlkwa{function}\hlstd{(}\hlkwc{par}\hlstd{,} \hlkwc{data}\hlstd{)\{}\hlcom{#calculates binomial MOM Estimator}
  \hlcom{#adapted from Prof. Cipolli's Chapter 7 notes}
  \hlstd{n} \hlkwb{<-} \hlstd{par[}\hlnum{1}\hlstd{]} \hlcom{#par[1] has the size n}
  \hlstd{p} \hlkwb{<-} \hlstd{par[}\hlnum{2}\hlstd{]} \hlcom{# par[2] has the probability p}
  \hlcom{#Here, k = 2 since estimator is highly variable at k>2}
  \hlcom{#First two population moments}
  \hlstd{EX1} \hlkwb{<-} \hlstd{n}\hlopt{*}\hlstd{p}
  \hlstd{EX2} \hlkwb{<-} \hlstd{n}\hlopt{*}\hlstd{p}\hlopt{*}\hlstd{(}\hlnum{1}\hlopt{-}\hlstd{p} \hlopt{+} \hlstd{n}\hlopt{*}\hlstd{p)} \hlcom{#moments found in notes}
  \hlstd{eq1} \hlkwb{<-} \hlstd{EX1} \hlopt{-} \hlkwd{mean}\hlstd{(data)} \hlcom{#sample moment 1}
  \hlstd{eq2} \hlkwb{<-} \hlstd{EX2} \hlopt{-} \hlkwd{mean}\hlstd{(data}\hlopt{^}\hlnum{2}\hlstd{)} \hlcom{#sample moment 2}
  \hlkwd{c}\hlstd{(eq1, eq2)}
\hlstd{\}}
\hlcom{# calculates Maximum Likelihood Estimator via negative log likelihood}
\hlstd{binom.MLE} \hlkwb{<-} \hlkwa{function}\hlstd{(}\hlkwc{par}\hlstd{,} \hlkwc{data}\hlstd{,} \hlkwc{neg} \hlstd{= T)\{}
  \hlcom{#adapted from Prof. Cipolli's Chapter 7 notes}
  \hlstd{n} \hlkwb{<-} \hlstd{par[}\hlnum{1}\hlstd{]}
  \hlstd{p} \hlkwb{<-} \hlstd{par[}\hlnum{2}\hlstd{]}
  \hlcom{#sums up the probability mass function for the sample }
  \hlstd{MLE} \hlkwb{<-} \hlkwd{sum}\hlstd{(}\hlkwd{dbinom}\hlstd{(}\hlkwc{x} \hlstd{= data,} \hlkwc{size} \hlstd{= n,} \hlkwc{prob} \hlstd{= p,} \hlkwc{log}\hlstd{=}\hlnum{TRUE}\hlstd{))}
  \hlkwd{ifelse}\hlstd{(}\hlopt{!}\hlstd{neg,MLE,}\hlopt{-}\hlstd{MLE)} \hlcom{#just in case neg is changed }
\hlstd{\}}

\hlkwd{library}\hlstd{(nleqslv)}\hlcom{# used to calculate MOM }

\hlstd{find.binom.mom.mle} \hlkwb{<-}\hlkwa{function}\hlstd{(}\hlkwc{n}\hlstd{,} \hlkwc{par}\hlstd{)\{}
  \hlcom{#function that calculates and plots MOM and MLE}
\hlstd{Sample} \hlkwb{=} \hlkwd{rbinom}\hlstd{(n,} \hlkwc{size} \hlstd{= par[}\hlnum{1}\hlstd{],} \hlkwc{prob} \hlstd{=par[}\hlnum{2}\hlstd{])}
\hlcom{#creating the sample for this set of parameters}
\hlstd{moms}\hlkwb{<-}\hlkwd{nleqslv}\hlstd{(}\hlkwc{x} \hlstd{=} \hlkwd{c}\hlstd{(par[}\hlnum{1}\hlstd{], par[}\hlnum{2}\hlstd{]),} \hlcom{#passes in size and probability}
\hlkwc{fn} \hlstd{= binom.mom,}
\hlkwc{data}\hlstd{=Sample)}
\hlcom{#this essentially minimizes the difference between the sample and population }
\hlcom{#moments. It then returns the corresponding n and p}
\hlstd{mles} \hlkwb{<-} \hlkwd{optim}\hlstd{(}\hlkwc{fn} \hlstd{= binom.MLE,}
              \hlkwc{par} \hlstd{=} \hlkwd{c}\hlstd{(par[}\hlnum{1}\hlstd{], par[}\hlnum{2}\hlstd{]),}
              \hlkwc{data} \hlstd{= Sample)}
\hlcom{#mles optimizes the MLE for the given parameters}
\hlcom{#plotting MOM graph}
  \hlstd{plot_mom} \hlkwb{=} \hlkwd{ggplot}\hlstd{()} \hlopt{+}
    \hlkwd{geom_histogram}\hlstd{(}\hlkwd{aes}\hlstd{(}\hlkwc{x} \hlstd{= Sample,}
                       \hlkwc{y} \hlstd{= ..density..),}
                   \hlkwc{binwidth} \hlstd{=} \hlnum{1}\hlstd{,}
                   \hlkwc{color} \hlstd{=} \hlstr{"black"}\hlstd{,}
                   \hlkwc{fill} \hlstd{=}\hlstr{"lightgreen"}\hlstd{)} \hlopt{+}
    \hlkwd{geom_hline}\hlstd{(}\hlkwc{yintercept} \hlstd{=} \hlnum{0}\hlstd{)} \hlopt{+}
    \hlkwd{theme_bw}\hlstd{()} \hlopt{+}
    \hlkwd{xlim}\hlstd{(}\hlnum{0}\hlstd{, par[}\hlnum{1}\hlstd{])} \hlopt{+}
    \hlkwd{geom_linerange}\hlstd{(}\hlkwc{size} \hlstd{=} \hlnum{0.7}\hlstd{,} \hlkwd{aes}\hlstd{(}\hlkwc{x}\hlstd{=}\hlnum{0}\hlopt{:}\hlstd{par[}\hlnum{1}\hlstd{],}
                       \hlkwc{ymin}\hlstd{=}\hlnum{0}\hlstd{,}
                       \hlkwc{ymax}\hlstd{=}\hlkwd{dbinom}\hlstd{(}\hlnum{0}\hlopt{:}\hlstd{par[}\hlnum{1}\hlstd{],}
                                   \hlkwc{size} \hlstd{=} \hlkwd{round}\hlstd{(moms}\hlopt{$}\hlstd{x[}\hlnum{1}\hlstd{]),}
                                   \hlkwc{prob} \hlstd{= moms}\hlopt{$}\hlstd{x[}\hlnum{2}\hlstd{])))} \hlopt{+}
    \hlkwd{xlab}\hlstd{(}\hlstr{"Number of successes"}\hlstd{)} \hlopt{+}
    \hlkwd{ylab}\hlstd{(}\hlstr{"Proportion"}\hlstd{)} \hlopt{+}
    \hlkwd{ggtitle}\hlstd{(}\hlkwd{paste}\hlstd{(}\hlstr{"Methods of Moments Estimator n ="}\hlstd{, n,} \hlkwc{sep}\hlstd{=}\hlstr{" "}\hlstd{),}
            \hlkwc{subtitle} \hlstd{=} \hlkwd{paste}\hlstd{(}\hlstr{"Size Estimate ="}\hlstd{,} \hlkwd{round}\hlstd{(moms}\hlopt{$}\hlstd{x[}\hlnum{1}\hlstd{],} \hlnum{3}\hlstd{),}
                             \hlstr{", Probability Estimate ="}\hlstd{,} \hlkwd{round}\hlstd{(moms}\hlopt{$}\hlstd{x[}\hlnum{2}\hlstd{],} \hlnum{3}\hlstd{),}
                             \hlkwc{sep} \hlstd{=}\hlstr{" "}\hlstd{))} \hlopt{+}
    \hlkwd{theme}\hlstd{(}\hlkwc{plot.title} \hlstd{=} \hlkwd{element_text}\hlstd{(}\hlkwc{hjust} \hlstd{=} \hlnum{0.5}\hlstd{,} \hlkwc{size} \hlstd{=} \hlnum{25}\hlstd{),}
          \hlkwc{plot.subtitle} \hlstd{=} \hlkwd{element_text}\hlstd{(}\hlkwc{hjust} \hlstd{=} \hlnum{0.5}\hlstd{,} \hlkwc{size} \hlstd{=} \hlnum{20}\hlstd{),}
          \hlkwc{axis.text}\hlstd{=}\hlkwd{element_text}\hlstd{(}\hlkwc{size}\hlstd{=}\hlnum{20}\hlstd{),}
          \hlkwc{axis.title.x} \hlstd{=} \hlkwd{element_text}\hlstd{(}\hlkwc{size} \hlstd{=} \hlnum{20}\hlstd{),}
          \hlkwc{axis.title.y} \hlstd{=} \hlkwd{element_text}\hlstd{(}\hlkwc{size} \hlstd{=} \hlnum{20}\hlstd{))}

\hlcom{#plotting MLE graph}
  \hlstd{plot_mle} \hlkwb{=} \hlkwd{ggplot}\hlstd{()} \hlopt{+}
    \hlkwd{geom_histogram}\hlstd{(}\hlkwd{aes}\hlstd{(}\hlkwc{x} \hlstd{= Sample,}
                       \hlkwc{y} \hlstd{= ..density..),}
                   \hlkwc{binwidth} \hlstd{=} \hlnum{1}\hlstd{,}
                   \hlkwc{color} \hlstd{=} \hlstr{"black"}\hlstd{,}
                   \hlkwc{fill} \hlstd{=}\hlstr{"lightgreen"}\hlstd{)} \hlopt{+}
    \hlkwd{geom_hline}\hlstd{(}\hlkwc{yintercept} \hlstd{=} \hlnum{0}\hlstd{)} \hlopt{+}
    \hlkwd{theme_bw}\hlstd{()} \hlopt{+}
    \hlkwd{xlim}\hlstd{(}\hlnum{0}\hlstd{, par[}\hlnum{1}\hlstd{])} \hlopt{+}
    \hlkwd{geom_linerange}\hlstd{(}\hlkwc{size} \hlstd{=} \hlnum{0.7}\hlstd{,}\hlkwd{aes}\hlstd{(}\hlkwc{x}\hlstd{=}\hlnum{0}\hlopt{:}\hlstd{par[}\hlnum{1}\hlstd{],}
                       \hlkwc{ymin}\hlstd{=}\hlnum{0}\hlstd{,}
                       \hlkwc{ymax}\hlstd{=}\hlkwd{dbinom}\hlstd{(}\hlnum{0}\hlopt{:}\hlstd{par[}\hlnum{1}\hlstd{],}
                                   \hlkwc{size} \hlstd{=} \hlkwd{round}\hlstd{(mles}\hlopt{$}\hlstd{par[}\hlnum{1}\hlstd{]),}
                                   \hlkwc{prob} \hlstd{= mles}\hlopt{$}\hlstd{par[}\hlnum{2}\hlstd{])))} \hlopt{+}
    \hlkwd{xlab}\hlstd{(}\hlstr{"Number of successes"}\hlstd{)} \hlopt{+}
    \hlkwd{ylab}\hlstd{(}\hlstr{"Proportion"}\hlstd{)} \hlopt{+}
    \hlkwd{ggtitle}\hlstd{(}\hlkwd{paste}\hlstd{(}\hlstr{"Maximum Likelihood Estimator n ="}\hlstd{, n,} \hlkwc{sep}\hlstd{=}\hlstr{" "}\hlstd{),}
            \hlkwc{subtitle} \hlstd{=} \hlkwd{paste}\hlstd{(}\hlstr{"Size Estimate ="}\hlstd{,} \hlkwd{round}\hlstd{(mles}\hlopt{$}\hlstd{par[}\hlnum{1}\hlstd{],} \hlnum{3}\hlstd{),}
                             \hlstr{", Probability Estimate ="}\hlstd{,} \hlkwd{round}\hlstd{(mles}\hlopt{$}\hlstd{par[}\hlnum{2}\hlstd{],} \hlnum{3}\hlstd{),}
                             \hlkwc{sep} \hlstd{=}\hlstr{" "}\hlstd{))} \hlopt{+}
    \hlkwd{theme}\hlstd{(}\hlkwc{plot.title} \hlstd{=} \hlkwd{element_text}\hlstd{(}\hlkwc{hjust} \hlstd{=} \hlnum{0.5}\hlstd{,} \hlkwc{size} \hlstd{=} \hlnum{25}\hlstd{),}
          \hlkwc{plot.subtitle} \hlstd{=} \hlkwd{element_text}\hlstd{(}\hlkwc{hjust} \hlstd{=} \hlnum{0.5}\hlstd{,} \hlkwc{size} \hlstd{=} \hlnum{20}\hlstd{),}
          \hlkwc{axis.text}\hlstd{=}\hlkwd{element_text}\hlstd{(}\hlkwc{size}\hlstd{=}\hlnum{20}\hlstd{),}
          \hlkwc{axis.title.x} \hlstd{=} \hlkwd{element_text}\hlstd{(}\hlkwc{size} \hlstd{=} \hlnum{20}\hlstd{),}
          \hlkwc{axis.title.y} \hlstd{=} \hlkwd{element_text}\hlstd{(}\hlkwc{size} \hlstd{=} \hlnum{20}\hlstd{))}


      \hlstd{plot_mom}\hlopt{+}\hlstd{plot_mle}
\hlstd{\}}
\end{alltt}
\end{kframe}
\end{knitrout}
  
\begin{knitrout}
\definecolor{shadecolor}{rgb}{0.969, 0.969, 0.969}\color{fgcolor}\begin{kframe}
\begin{alltt}
 \hlkwd{find.binom.mom.mle}\hlstd{(}\hlnum{10}\hlstd{,} \hlkwd{c}\hlstd{(t1, prob1))}
\end{alltt}
\end{kframe}
\includegraphics[width=\maxwidth]{figure/unnamed-chunk-3-1} 
\begin{kframe}\begin{alltt}
 \hlkwd{find.binom.mom.mle}\hlstd{(}\hlnum{10}\hlstd{,} \hlkwd{c}\hlstd{(t2, prob2))}
\end{alltt}
\end{kframe}
\includegraphics[width=\maxwidth]{figure/unnamed-chunk-3-2} 
\end{knitrout}
  %%%%%%%%%%%%%%%%%%%%%%%%%%%%%%%%%%%%%%%%%%%%%%%%%%%%%%%%%%%%%%%%%%%%%%%%%%%%%%%%%%%%%%%%%%%
  %%%%%%%%%  Part (d)
  %%%%%%%%%%%%%%%%%%%%%%%%%%%%%%%%%%%%%%%%%%%%%%%%%%%%%%%%%%%%%%%%%%%%%%%%%%%%%%%%%%%%%%%%%%%
  \item Generate a random sample of size $n=25$ for your two sets of parameter(s). 
  Calculate the method of moments estimator(s) and maximum likelihood estimator(s).
  In a $1 \times 2$ grid, plot a histogram (with bin size 1) of each set of data 
  with (1) the method of moments estimated distribution, (2) the maximum likelihood 
  estimated distribution, and superimpose the true distribution in both.\\
  \textbf{Solution:}
\begin{knitrout}
\definecolor{shadecolor}{rgb}{0.969, 0.969, 0.969}\color{fgcolor}\begin{kframe}
\begin{alltt}
\hlkwd{find.binom.mom.mle}\hlstd{(}\hlnum{25}\hlstd{,} \hlkwd{c}\hlstd{(t1, prob1))}
\end{alltt}
\end{kframe}
\includegraphics[width=\maxwidth]{figure/unnamed-chunk-4-1} 
\begin{kframe}\begin{alltt}
\hlkwd{find.binom.mom.mle}\hlstd{(}\hlnum{25}\hlstd{,} \hlkwd{c}\hlstd{(t2, prob2))}
\end{alltt}
\end{kframe}
\includegraphics[width=\maxwidth]{figure/unnamed-chunk-4-2} 
\end{knitrout}
  %%%%%%%%%%%%%%%%%%%%%%%%%%%%%%%%%%%%%%%%%%%%%%%%%%%%%%%%%%%%%%%%%%%%%%%%%%%%%%%%%%%%%%%%%%%
  %%%%%%%%%  Part (e)
  %%%%%%%%%%%%%%%%%%%%%%%%%%%%%%%%%%%%%%%%%%%%%%%%%%%%%%%%%%%%%%%%%%%%%%%%%%%%%%%%%%%%%%%%%%%
  \item Generate a random sample of size $n=100$ for your two sets of parameter(s).
  Calculate the method of moments estimator(s) and maximum likelihood estimator(s). 
  In a $1 \times 2$ grid, plot a histogram (with bin size 1) of each set of data 
  with (1) the method of moments estimated distribution, (2) the maximum likelihood
  estimated distribution, and superimpose the true distribution in both.\\
  \textbf{Solution:}
\begin{knitrout}
\definecolor{shadecolor}{rgb}{0.969, 0.969, 0.969}\color{fgcolor}\begin{kframe}
\begin{alltt}
\hlkwd{find.binom.mom.mle}\hlstd{(}\hlnum{100}\hlstd{,} \hlkwd{c}\hlstd{(t1, prob1))}
\end{alltt}
\end{kframe}
\includegraphics[width=\maxwidth]{figure/unnamed-chunk-5-1} 
\begin{kframe}\begin{alltt}
\hlkwd{find.binom.mom.mle}\hlstd{(}\hlnum{100}\hlstd{,} \hlkwd{c}\hlstd{(t2, prob2))}
\end{alltt}
\end{kframe}
\includegraphics[width=\maxwidth]{figure/unnamed-chunk-5-2} 
\end{knitrout}
  %%%%%%%%%%%%%%%%%%%%%%%%%%%%%%%%%%%%%%%%%%%%%%%%%%%%%%%%%%%%%%%%%%%%%%%%%%%%%%%%%%%%%%%%%%%
  %%%%%%%%%  Part (f)
  %%%%%%%%%%%%%%%%%%%%%%%%%%%%%%%%%%%%%%%%%%%%%%%%%%%%%%%%%%%%%%%%%%%%%%%%%%%%%%%%%%%%%%%%%%%
  \item Generate a random sample of size $n=100$ for your two sets of parameter(s).
  Calculate the method of moments estimator(s) and maximum likelihood estimator(s).
  In a $1 \times 2$ grid, plot a histogram (with bin size 1) of each set of data 
  with (1) the method of moments estimated distribution, (2) the maximum likelihood
  estimated distribution, and superimpose the true distribution in both.\\
  \textbf{Solution:}
\begin{knitrout}
\definecolor{shadecolor}{rgb}{0.969, 0.969, 0.969}\color{fgcolor}\begin{kframe}
\begin{alltt}
\hlkwd{find.binom.mom.mle}\hlstd{(}\hlnum{1000}\hlstd{,} \hlkwd{c}\hlstd{(t1, prob1))}
\end{alltt}
\end{kframe}
\includegraphics[width=\maxwidth]{figure/unnamed-chunk-6-1} 
\begin{kframe}\begin{alltt}
\hlkwd{find.binom.mom.mle}\hlstd{(}\hlnum{1000}\hlstd{,} \hlkwd{c}\hlstd{(t2, prob2))}
\end{alltt}
\end{kframe}
\includegraphics[width=\maxwidth]{figure/unnamed-chunk-6-2} 
\end{knitrout}
  %%%%%%%%%%%%%%%%%%%%%%%%%%%%%%%%%%%%%%%%%%%%%%%%%%%%%%%%%%%%%%%%%%%%%%%%%%%%%%%%%%%%%%%%%%%
  %%%%%%%%%  Part (g)
  %%%%%%%%%%%%%%%%%%%%%%%%%%%%%%%%%%%%%%%%%%%%%%%%%%%%%%%%%%%%%%%%%%%%%%%%%%%%%%%%%%%%%%%%%%%
  \item Comment on the results of parts (c)-(f). \\
  \textbf{Solution:}
  As seen in the plots above, the size estimate for the MOM Estimator is very inaccurate 
  for n = 10 and n = 25, but gets increasingly closer as n increases. This corresponds with 
  the Weak Law of Large Numbers. This pattern is also seen for the MOM Estimator of the 
  probability estimate. The MLE estimator is a different since the size estimate provided by 
  it is almost always completely accurate, and the probability estimate is very close even for 
  n = 10. To find out which Estimator is better, we would need to run this for a larger array 
  of samples, since it is completely possible that the MLE is not as good at providing estimate 
  for some other parameters. Overall, these estimates look very close to the data generates, 
  especially at higher n values. 
\end{enumerate}
\end{enumerate}%End overall enumerate
\newpage
\bibliography{bib}
\end{document}
